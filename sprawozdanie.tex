\documentclass[a4paper]{article}
\usepackage[left=1cm,right=1cm,top=3cm,bottom=3cm]{geometry}
\usepackage[T1]{fontenc}
\usepackage{amsmath}
\usepackage{mathtools}
\usepackage{amssymb}
\usepackage{indentfirst}

\title{\textbf{Pracownia z Analizy Numerycznej}\\{\Large Sprawozdanie do zadania P1.7}}
\author{Mateusz Leonowicz}

\begin{document}
\maketitle

\section{Wstęp}
    Wiele problemów w matematyce, fizyce czy informatyce sprowadzić można do wyznaczenia miejsc
    zerowych danego równania algebraicznego. Często nie wystarczy nam prosta analiza funkcji, a
    jedyne co możemy zrobić, to obliczenie jej wartości w danej, skończonej liczbie punktów jej dziedziny.
    Dlatego temat ten stał się jednym z fundamentalnych zagadnień analizy numerycznej, dzięki czemu
    powstało wiele metod iteracyjnych, które oczywiście mają swoje zalety i wady.
    
    \vspace{5mm}

    Celem tego sprawozadnie jest przedstawienie metod obliczania pierwiastków wielomianów w postaci
    \[ax^3 + bx^2 + cx + d = 0 \qquad a, b, c, d \in \mathbb{R}\]
    które pozwalają komputerom na uzyskanie wyników z kontrolowanym błędem. Przedstawię trzy metody numeryczne oraz 
    rozwiązania z użyciem wzorów Cardano. Opiszę ich działadnie i charakterystykę, a następnie umieszczę ich porównanie
    razem z podsumowaniem.

    Wszystkie testy przeprowadzane będą z użyciem języka do analizy numerycznej Julia.
    
\tableofcontents
\newpage

\section{Metoda Newtona}


\end{document}
