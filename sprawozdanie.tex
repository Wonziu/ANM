\documentclass[a4paper]{article}
\usepackage[left=2cm,right=2cm,top=3cm,bottom=3cm]{geometry}
\usepackage[T1]{fontenc}
\usepackage{amsmath}
\usepackage{mathtools}
\usepackage{amssymb}
\usepackage{indentfirst}

\title{\textbf{Pracownia z Analizy Numerycznej}\\{\Large Sprawozdanie do zadania P1.7}}
\author{Mateusz Leonowicz}

\begin{document}
\maketitle

\section{Wstęp}
    Wiele problemów w matematyce, fizyce czy informatyce sprowadzić można do wyznaczenia miejsc
    zerowych danego równania algebraicznego. Często nie wystarczy nam prosta analiza funkcji, a
    jedyne co możemy zrobić, to obliczenie jej wartości w danej, skończonej liczbie punktów jej dziedziny.
    Dlatego temat ten stał się jednym z fundamentalnych zagadnień analizy numerycznej, dzięki czemu
    powstało wiele metod iteracyjnych, które oczywiście mają swoje zalety i wady.
    
    \vspace{5mm}

    Celem tego sprawozdania jest przedstawienie metod obliczania pierwiastków wielomianów w postaci
    \[ax^3 + bx^2 + cx + d = 0 \qquad a, b, c, d \in \mathbb{R}\]
    które pozwalają komputerom na uzyskanie wyników z kontrolowanym błędem. Przedstawię trzy metody numeryczne oraz 
    rozwiązania z użyciem wzorów Cardano. Opiszę ich działadnie i charakterystykę, a następnie umieszczę ich porównanie
    razem z podsumowaniem.

    Wszystkie testy przeprowadzane będą z użyciem języka do analizy numerycznej Julia.
    
\tableofcontents
\newpage

\section{Metoda Newtona}
    Niech $f(x)$ będzie funkcją, której miejsce zerowe chcemy wyznaczyć. Niech $\alpha$ będzie takim zerem, a $x$
    jego przybliżeniem. Z twierdzenia Taylora, wiemy, że przybliżenie funkcji $f$, możemy zapisać w postaci:
    \[
        0 = f(\alpha) = f(x + e) = f(x) + ef'(x) + \frac{f''(\xi)}{2!}e^2 \qquad \xi \in \text{interv}(x, \alpha) 
    \tag{1}\]
    Jeśli nasz wyraz $\frac{f''(\xi)}{2!}e^2$ będzie dostatecznie mały, to możemy go pominąć i rozwiązać równanie 
    względem $e$, co daje nam:
    \[
        e = \frac{-f(x)}{f'(x)}  
    \]
    Jeśli $x$ jest dostatecznie dobrym przybliżeniem $\alpha$, to $x - e$ będzie jeszcze lepszym przybliżeniem tego
    pierwiastka. Na tej różnicy opiera się metoda Newtona, która po wybraniu startowego przybliżenia $x_0$ zera $\alpha$
    polega na stosowaniu rekurencyjnego wzoru:
    \[
        x_{n+1} = x_n - \frac{f(x)}{f'(x)} \qquad n \geq 0
    \tag{2}\]
    Dobranie startowego punktu $x_0$ w tej metodzie jest bardzo istotne, gdyż metoda Newtona nie zawsze jest zbieżna.
    Jeśli dobierzemy odpowiednio punkt początkowy, to będziemy mieli zbieżność kwadratową (z wyjątkiem przypadków, gdy 
    istnieją wielokrotne zera funkcji. Wtedy otrzymamy zbieżność liniową).
    \vspace{5mm}
    Metoda Newtona.

\end{document}
